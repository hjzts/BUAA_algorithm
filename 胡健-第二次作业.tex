\documentclass{article}

% Language setting
% Replace `English' with e.g. `Spanish' to change the document language
\usepackage[english]{babel}
\usepackage{ctex}

% Set page size and margins
% Replace `letter paper' with`a4paper' for UK/EU standard size
\usepackage[a4paper,top=1.5cm,bottom=1.5cm,left=2cm,right=2cm,marginparwidth=1.75cm]{geometry}

% Useful packages
\usepackage{amsmath}
\usepackage{amssymb}
% 引入{amssymb}包,避免输入\nexists(存在否定)这个符号时 报未定义符号的错误
\usepackage{graphicx}
\usepackage{listings}
\usepackage[ruled]{algorithm2e}
\usepackage[colorlinks=true, allcolors=blue]{hyperref}
\usepackage{algpseudocode}
\usepackage{ dsfont }
% \usepackage{algorithm}

\SetKwProg{Function}{function}{:}{end}

% TODO 个人信息参数全局变量
% -------------- define personal variables begin --------------
\newcommand\myCollege{}
\newcommand\myId{22373058}
\newcommand\myName{胡健}
\newcommand\myHomeworkId{1}
% --------------  define personal variables end  --------------

\title{\heiti \myCollege《算法设计与分析》第\myHomeworkId 次作业} % 文章标题
\author{\myCollege \quad \myId \quad \myName} % 作者信息

\begin{document}
\maketitle

\section{不完美字符串 (20 分)}

对于一个长度为 $n$ 的仅由 0 和 1 构成的字符串 s ,其不完美度定义为所有相邻字符对中互
不相同的字符对数量的和。比如,长度为 4 的字符串 “0010” 的不完美度为 2,这是由 01 和 10 两个相邻数对贡献得到的。形式化地,字符串 $s$ 的不完美度如下式:
\begin{align*}
    & f(s) =  \sum_{i=1}^{n-1} [s_i \ne s_{i+1}]
\end{align*}
([] 运算的结果取决于其中布尔表达式的真值。若为假,其值为 0,若为真,其值为 1)

现给定一个长度为 n 的字符串 s,每一位仅由 0、 1 和 ? 组成。你需要将其中的每一个 ? 替
换成 0 或者 1 ,且最小化字符串 s 的不完美度 $f(s)$ 。

例如,给定的字符串为 “?1?0”,一种使不完美度最小的替换方案为 “1110”,其不完美度为 1,
可以证明,没有其他替换方案,

请使用动态规划算法求解最小的不完美度,并给出一个替换方案。请描述算法的核心思想,
给出算法伪代码并分析其对应的时间复杂度。

\subsection{题目分析与核心思想}

\subsection{算法与伪代码}

\subsubsection{基于排序算法}
\begin{algorithm}[H]

\caption{基于归并排序处理游戏获奖问题}
\LinesNumbered
\KwIn{正整数$n$,为数组长度,每名选手的得分数组:$A[1...n]$}
\KwOut{一个整数Ans,表示主办方发放的奖金}

\Function{$main(n,A)$}{
    $\text{MergeSort}(A, 1, n)$\\
    $Ans \leftarrow 0 $\\
    \For {$i \leftarrow 1 \ to \ \lfloor \frac{n}{3} \rfloor$} {
        $Ans += A[i]$
    }
    \Return{$Ans$}
}
\end{algorithm}


\subsection{时间复杂度分析}

% \newpage

\section{鲜花组合问题 (20 分)}

花店共有 $n$ 不同颜色的花,其中第 $i$ 种库存有 $a_i$ 枝,现要从中选出$m$ 枝花组成一束鲜花。

请设计算法计算有多少种组合一束花的方案,请描述算法的核心思想,给出算法伪代码并
分析其对应的时间复杂度。(两种方案不同当且仅当存在一个花的种类 $i$,两种方案中第 $i$ 种花的数量不同)

\subsection{题目分析与核心思想}

\subsection{算法与伪代码}

\subsection{时间复杂度分析}

% \newpage

\section{最长公共上升子序列 (20 分)}

对两个序列 $A$ 和 $B$ ,序列 $s$ 为 $A$ 和 $B$ 的公共上升子序列,当且仅当 $s$ 是 $A$ 和 $B$ 的公共子序列,且 $s$ 是上升子序列($s_i < s_{i+1}, \forall \ 1 \le i < |s|$)。

对两个序列 $A$ 和 $B$ ,若某个序列 $s$ 是 $A$ 和 $B$ 的公共上升子序列,且对于任意的 $A$ 和 $B$ 的公共上升子序列 $t$,都有 $|t| \le |s|$,那么 $s$ 称为 $A$ 和 $B$ 的最长公共上升子序列。

例如,给定两个序列 < 2, 3, 1, 6, 5, 4, 6 > 和 < 1, 3, 5, 6 >,其一个最长公共上升子序列为< 3, 5, 6 >。

给定两个长度为 $n$ 的序列 $A$ 和 $B$。请设计一个动态规划算法,求它们的最长公共上升子序列的长度。请描述算法的核心思想,给出算法伪代码并分析其对应的时间复杂度。

\subsection{题目分析与核心思想}

\subsection{算法与伪代码}

\subsection{时间复杂度分析}

% \newpage

\section{叠塔问题 (20 分)}

给定 $n$ 块积木,编号为 $1$ 到 $n$。第 $i$ 块积木的重量为 $w_i$($w_i$ 为整数),硬度为 $s_i$,价值为 $v_i$。   

现要从中选择部分积木垂直摞成一座塔,要求每块积木满足如下条件:

若第 $i$ 块积木在积木塔中,那么在其之上摆放的所有积木的重量之和不能超过第 $i$ 块积木的
硬度。

试设计算法求出满足上述条件的价值和最大的积木塔,输出摆放方案和最大价值和。请描述算法的核心思想,给出算法伪代码并分析其对应的时间复杂度。

\subsection{题目分析与核心思想}

\subsection{算法与伪代码}

\subsection{时间复杂度分析}

% \newpage

\section{最小划分问题 (20 分)}

对一个序列 $a$ 上的某两个数 $a_i$ 和 $a_j$ ,若 $i < j$ 且 $a_i \ne a_j$ ,则称 $(i, j)$ 为一个不同数对。一个序列 $a$ 的不同数对数为序列中所有不同数对的个数之和。例如,序列 < 1, 0, 1, 0 > 的不同数对有 (1, 2), (1, 4), (2, 3), (3, 4) 四个。

给你一个长度为 $n$ 的正整数序列 $a$ 和一个正整数 $k$,满足序列中的任意一个数 $a_i \in [1, n]$ 。请你将其划分为 $k$ 个子段,最小化每个子段最小不同数对数相加的和。你只需要回答这个最小
的和。

例如,给定序列为 < 1, 1, 3, 3, 3, 2, 1 > 和参数 $k = 3$,则答案为 1,对应的一个划分方法为< 1, 1 >, < 3, 3, 3 >, < 2, 1 > 。

请使用动态规划求解该问题,描述算法的核心思想,给出算法伪代码并分析其对应的时间复杂度。

\subsection{题目分析与核心思想}

\subsection{算法与伪代码}

\subsection{时间复杂度分析}

\end{document}
